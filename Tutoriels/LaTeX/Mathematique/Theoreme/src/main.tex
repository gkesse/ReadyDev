%================================================
\documentclass{book}
%================================================
\usepackage[utf8]{inputenc}
\usepackage[T1]{fontenc}
\usepackage[francais]{babel}
\usepackage{amsthm}
\usepackage{amsmath}
\usepackage{amssymb}
\usepackage{mathrsfs}
%================================================
\newtheorem*{petit_nom}{Lemme}
\newtheorem{petit_nom1}{Proposition}
\newtheorem{petit_nom2}{Définition}[chapter]
\newtheorem{petit_nom3}{Définition}[section]
%================================================
\begin{document}
%================================================
\chapter{Les ours}
%================================================
\chapter{Les lapins}
%================================================
\section{les lapins nains}
%================================================
\begin{petit_nom}
ils aiment les carottes
\end{petit_nom}
%================================================
\begin{petit_nom}[des lapins]
ils aiment les carottes
\end{petit_nom}
%================================================
\begin{petit_nom1}
ils aiment les carottes
\end{petit_nom1}
%================================================
\begin{petit_nom1}[des lapins]
ils aiment les carottes
\end{petit_nom1}
%================================================
\begin{petit_nom2}
ils aiment les carottes
\end{petit_nom2}
%================================================
\begin{petit_nom2}[des lapins]
ils aiment les carottes
\end{petit_nom2}
%================================================
\section{les autres}
%================================================
\begin{petit_nom3}
ils aiment les carottes
\end{petit_nom3}
%================================================
\begin{petit_nom3}[des lapins]
ils aiment les carottes
\end{petit_nom3}
%================================================
\end{document}
%================================================
