%================================================
\documentclass{article}
%================================================
\usepackage[utf8]{inputenc}
\usepackage[T1]{fontenc}
\usepackage[francais]{babel}
\usepackage{amsmath}
%================================================
\begin{document}
%================================================
\parindent=0em
\textbf{Mathématique Trigonométrie} \\ \\
%================================================
\textbf{Cosinus} \\ \\
%================================================
$$f(x)=\cos(x+1)$$ \\ \\
%================================================
\textbf{Sinus} \\ \\
%================================================
$$f(x)=\sin(x+1)$$ \\ \\
%================================================
\textbf{Tangente} \\ \\
%================================================
$$f(x)=\tan(x+1)$$ \\ \\
%================================================
\textbf{Arc Cosinus} \\ \\
%================================================
$$f(x)=\arccos(x+1)$$ \\ \\
%================================================
\textbf{Arc Sinus} \\ \\
%================================================
$$f(x)=\arcsin(x+1)$$ \\ \\
%================================================
\textbf{Arc Tangente} \\ \\
%================================================
$$f(x)=\arctan(x+1)$$ \\ \\
%================================================
\textbf{Cosinus Hyberbolique} \\ \\
%================================================
$$f(x)=\cosh(x+1)$$ \\ \\
%================================================
\textbf{Sinus Hyberbolique} \\ \\
%================================================
$$f(x)=\sinh(x+1)$$ \\ \\
%================================================
\textbf{Tangente Hyberbolique} \\ \\
%================================================
$$f(x)=\tanh(x+1)$$ \\ \\
%================================================
\end{document}
%================================================
